%-------------------------------------------------------------------------------
%	SECTION TITLE
%-------------------------------------------------------------------------------
\cvsection{Kunder og Prosjekter}


%-------------------------------------------------------------------------------
%	CONTENT
%-------------------------------------------------------------------------------
\begin{cventries}
%---------------------------------------------------------
    \kunde
    {Sparebank1} % Kunde
    \project
    {Sparing}
    {2018 - }
    {Jobbet innenfor sparedomenet. Løsninger for fond, aksjer bankkontoer, sparemål og sparerådgivare.}
    {React, Redux, Typescript, Kotlin, Java, HTML, Accessibility, Less, Jenkins, Jira, Confluence}

%---------------------------------------------------------
    \kunde
    {Toll} % Kunde
    \project
    {Ekpressfortolling}
    {2017 - 2018}
    {Jag var lead på frontend.
    Løsningen er et system som muliggjør for transportselskaper å kjøre over grensen fra Sverige uten stopp for kontrollering av
    papirer og last dersom alle papirer er sendt inn på forhånd. Restbasert mikrotjenestearkitektur med en klient for selvbetjening og en intern for tollerne. }
    {React, Redux, PostCSS, Karma,Chai, Sinon, webpack, yarn, Jira, Confluence, Docker, Jenkins, Spring Boot, HATEOAS}
    \project
    {Ny brukerfalte}
    {2017}
    {Løsningen besto av en micro-frontend arkitektur med forskjellige registre over feks.
    kunder, myndigheter, poststeder og land. Kom inn på prosjektet efter anbefaling.
    Hovedansvaret var frontend men utviklet også selvstendig hele mikrotjenstens backend.
    Jeg var også involvert i DevOps med Openshift, Docker, Jenkins Pipelines og automatisering mot Jira.}
    {React, Redux, PostCSS, Karma,Chai, Sinon, webpack, yarn, Jira, Confluence, Docker, Jenkins, Spring Boot, HATEOAS}

%---------------------------------------------------------

    \kunde
    {Gjensidige Forsikring} % Kunde
    \project
    {Bable}
    {2016}
    {En chat bot hvis førmål var å gjøre den ellers kompliserte kjøpsprosessen for bilforsikring enklere, spesielt for unge. Løsningen best av en React/Redux-app med Enonic CMS som datakilde.}
    {React, Redux, Less, Jasmine, webpack, Jira, Enonic CMS}
    \project
    {Kundeutbytte}
    {2014 - 2016}
    {Utvikling av løsningen for utbetaling av kundeutbytte. Var med 2 ganger (2014, 2016). Første gangen skulle løsningen over på nytt design. Andre gangen ble jeg ønsket tilbake av gjensidige-stiftelsen. Jeg tok initiativ til å få skrive om løsningen fra XSLT til React/Redux, og dette ble akseptert. Kodebasen er nå moderne og testbar.}
    {React, Redux, Less, Jasmine, webpack, Jira, Enonic CMS, XSLT}
    \project
    {Ny Nettbutikk}
    {2015}
    {Omskrivning av løsning også skrevet i AngularJS. Løsningen var utviklet i India og kvaliteten var middelmådig så en omskrivning var nødvendig. Gjensidige satt sammen et team av erfarne utviklere, og løsningen ble omskrevet med fokus på kvalitet og testbarhet.}
    {AngularJS, LESS, Jasmine, Gulp, Jenkins}
    \project
    {Nytt Intranett}
    {2015}
    {Gammelt intranett på enonic cms skulle fornyes. Utvikling og design av nytt intranett for Gjensidige. Som arkitekt anbefalte jeg nytt CMS og ny arkitektur. Deltok som utvikler og rådgiver for utvikling av nytt intranett basert på Enonic XP. Etablerte et rammeverk for gjenbrukbare frontend-komponenter.}
    {Enonic XP, Enonic CMS, Thymeleaf, Javascript, Less, HTML}


%---------------------------------------------------------
    \kunde
    {Tine} % Kunde
    \project
    {Skolemelk.no}
    {2014}
    {Websiden skolemelk.no hadde en umoderne design, og benyttet gammel teknologi. TINE ønsket å gjøre en redesign av løsningen for å tilfredsstille dagens krav til funksjonalitet og design basert på moderne teknologi og rammeverk.}
    {Enonic CMS, XSLT, HTML, CSS, jQuery}
    \project
    {TINE-14}
    {2014}
    {Implementerte sidene for den nye produktserien TINE-14 i Enonics CMS. Frontend ble levert av en ekstern designbyrå. Sidene ble kopiert fra en ekstern side og etablert i Enonics CMS. Datakilder og logikk ble lagt til sidene.}
    {Enonic CMS, XSLT, HTML, CSS, jQuery, Grunt}

%---------------------------------------------------------
    \kunde
    {Norsk Tipping} % Kunde
    \project
    {Instaspill}
    {2013 - 2014}
    {Opprette nye nettsider for Norsk Tipping AS. Prosjektet omfattet arbeid i Enonic CMS. Kjerneteknologiene som ble brukt er XSLT, HTML, jQuery og less.}
    {Enonic CMS, XSLT, HTML, Less, jQuery}
%---------------------------------------------------------

    \kunde
    {Mesta} % Kunde
    \project
    {Driftslogg}
    {2012 - 2013}
    {Driftslogg er en webapplikasjon hvor Mesta AS kjøretøyoperatører rapporterer sitt arbeid}
    {Java, Spring MVC, Spring Security, JSP, jQuery}
%---------------------------------------------------------


    \kunde
    {EDB ErgoGroup} % Kunde
    \project
    {Esporing}
    {2011}
    {Utvikling av en sporingsløsning for matvarer. Løsningen er til for å kunne identifisere produkter helt tilbake til råvarekilden ved oppdagelse av smitte. Den er bygget i Java, med bruk av Spring.}
    {Java, Spring, JSP, jQuery, CSS, Jira}
%---------------------------------------------------------

\end{cventries}
